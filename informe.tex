\documentclass{article}
\usepackage[utf8]{inputenc}
\usepackage[spanish]{babel}
\usepackage{graphicx}

\title{Proyecto Sopa}
\author{Paola Olguín Muñoz}

\begin{document}
  \begin{titlepage}
    \maketitle
  \end{titlepage}
  \newpage
  \tableofcontents
  \listoffigures
  \newpage
  \section{Introducción}



    El proyecto nos pide el desarrollo de la  solución en Java, por lo que, de
    igual forma, se comentará un poco sobre el lenguaje antes de hablar sobre el
    algoritmos de solución como tal.

    \subsection{Programación Orientada a Objetos}
      Un paradigma se define como una forma de ver el mundo. En el ámbito
      informático tiene una relación directa con la clasificación de lenguajes según
      las características que tengan y el objetivo para el cual supuestamente debe
      usarse.

      Para este proyecto se usa el lenguaje Java, que está orientado a objetos.
      Para comprender mejor como funciona, nos basaremos en el uso de clases y objetos.
      Una clase es el concepto de una cosa, mientras que un objeto corresponde a la
      cosa como tal. Para llevarlo a una forma más cotidiana, una clase sería el
      equivalente a la palabra que se refiere a algo, mientras que el objeto sería
      la cosa tangible.

    \subsection{Java}
      El lenguaje de programación a utilizar es Java, que, como dicho antes, es
      orientado a objetos, por lo que funciona con clases y objetos. Además, es un
      lenguaje de propósito general que se hizo con la idea de que fuera transportable.
      Por lo anterior, un programa en Java se compila, pues una vez compilado puede
      ejecutarse en cualquier máquina, gracias a que no se traspasa el código
      directamente a código máquina (que sería específico al hardware), sino que,
      al compilar, Java pasa el código a una máquina virtual, la cual ahorra problemas
      con respecto a los sistemas operativos y hardware de cada computador.

    \subsection{Diagramas}
    Para mostrar el comportamiento y la lógica detrás del programa se incluirán
    diagramas de clase, flujo y secuencia.



  \section{Descripción del problema}

    Se nos pide resolver una sopa de letras, dada una y una lista de palabras que
    buscar en ella. Sin embargo, no es una sopa común, sino que tiene particularidades
    para formar las palabras dentro de ella.

    En resumen, se nos pide buscar unas ciertas palabras, y luego especificar si
    se encontraron o no en un archivo que creamos.

    \subsection{Consideraciones}
      A continuación se menciona las cosas a tener en cuenta para la programación
      de la solución
    \begin{enumerate}
      \item Una palabra puede formarse sólo por casillas adyacentes.
      \item Diagonales no cuentan como palabras.
      \item Las palabras pueden doblarse.
      \item Sólo se pregunta si la palabra está, no cuántas veces.
    \end{enumerate}



  \section{Descripción de la solución}
    \textit{¿Qué hacemos al buscar soluciones en un espacio definido?}

    Para responder la pregunta, nos ponemos, metafóricamente, en la situación.
    Imaginemos que estamos dentro de un laberinto, ¿cómo logramos salir? Como
    personas tenemos la desventaja que sólo podemos recorrer un camino, pero,
    ¿qué ocurriría si podemos recorrer todos a la vez?

    Lograríamos encontrar la solución mucho antes. Si a eso le agregamos que
    eliminamos de nuestras posibles soluciones cualquiera que tenga una pared inmediata,
    nos evitamos una gran cantidad de caminos que no nos llevan a la salida.

    Una idea parecida fue lo que guió la solución del problema que se nos planteó,
    la búsqueda de distintas palabras en una sopa de letras. Al igual que en el caso
    del laboratorio, tenemos un espacio reducido donde buscar, además con la ventaja
    que sabemos qué palabra estamos buscando.

    \subsection{Algoritmo de Resolución}

    Primero, por como se ingresa la sopa de letras y las palabras que buscamos,
    hay que leer dos archivos, uno de la sopa y otro de las palabras a buscar en
    ella. Para esto se tuvo en consideración dos formas; leer cada fila de los
    archivos a un arreglo (Array) o a un ArrayList.
    Se optó por la segunda, pues sus objetos tienen un tamaño variable, por lo que
    no es necesario leer los archivos dos veces o ir aumentando un array para guardar
    todos las lineas del archivo.


    Guardados ambos archivos, separamos la sopa en una matriz, tal que nos queda
    un ArrayList de arrays de Strings. De esta forma, para obetener una posición en
    y, recorremos el ArrayList, y para obtener la posición en x recorremos el array
    interior.
    Obtenemos entonces una forma de reconocer una posición (y,x).


    Antes de comenzar nos preguntamos qué buscamos. Si buscamos la palabra como tal,
    estaríamos buscando un string. Sin embargo, a un String no se le puede ver la
    posición de arriba. Por lo tanto, nos definimos un Estado, que contiene posiciones,
    pues con estas y la sopa ya podemos acceder al elemento en cada una. Agregamos
    además un atributo generación, que luego nos ayudará a ver si hemos encontrado
    la palabra o no. Para mayor comprensión, revisar \ref{fig: 1}

    \begin{figure}
      \centering
      \includegraphics{}
      \caption{}
      \label{fig: 1}

    \end{figure}

    Comenzamos a recorrer la sopa, buscando la primera letra de cada palabra. Al
    hallarla, agregamos su Estado a una lista de primeros Estados. Estos primeros
    Estados tienen un ArrayList con un solo elemento; que corresponde a la posición
    donde se halló la primera letra, y una generación igual a uno.
    En el caso que no se encuentre la primera letra se retorna un Estado inválido
    que contiene una generacion cero y una ArrayList vacío donde deberíamos tener
    posiciones.

    Encontrado el primer Estado, hay que revisar las casillas adyacentes, verificando
    que no nos salgamos de la matriz. En esta verificación se ve también si la siguiente
    letra coincide con la siguiente letra buscada en la palabra, pues si no lo hace,
    el Estado queda como estaba, sin agregar la siguiente posición. En cambio, si
    encontramos la letra que sigue, se guarda su nueva posición junto con la posición
    de todas las letras anteriores a ella, además de su generación. Esto continúa
    hasta recorrer todas las letras de todas las palabras y recorrer todas las
    posiciones adyacentes de la primera letra de cada una.

    Así, volviendo al punto anterior, tendríamos ArrayList con los Estados de las
    primeras posiciones, y a partir de cada uno de ellos, formaríamos los Estados
    siguientes, agregandolos a un ArrayList de Estados Abiertos. Estos Estados se
    recorren, verificando su generacion; si esta es igual al largo de la palabra,
    significa que hemos encontrado la palabra. Sin embargo, aún no sabemos si es
    válida esa respuesta, pues una de las condiciones de la solución es que la palabra
    no se puede doblar sobre si misma. Es decir, no puede repetir posiciones. Por
    lo mismo, si se encuentra un Estado con generacion igual al largo de la palabra,
    se revisan sus posiciones, tal que no repita ninguna.

    En el caso contrario, de que no se encuentre ningún Estado con generación del
    largo, y ya se recorrieron todas las palabras, sabemos que ningún estado cumplió,
    por lo cual, si bien se encontró la primera letra, en algún lado la palabra dejó
    de ser similar. Con a este tipo de Estado lo trabajamos como si no hubiera
    encontrado la primera letra, pues así separamos los casos en dos solamente;
    si la palabra se encontró o no.
    Finalmente, si la generacion es cero ya sabemos que la palabra definitivamente
    no se encontró, pues ni siquiera estaba la primera letra.

    De esa manera ya tenemos algo para entregar al usuario. Con el último análisis
    formamos un ArrayList de Estados Validos, el cual contiene Estados con generación
    cero o del largo de la palabra, sin casos intermedios. Luego se escribe en un
    archivo qué palabra se halló y cual no, a partir de la generación.

    \subsubsection{Consideraciones de la entrada}
    \begin{enumerate}
      \item Se debe tener cuidado con ambos archivos, que no tengan ningún salto de
      línea al final o al inicio.
      \item Las palabras se deben escibir de la misma forma que se buscarán en
      la sopa.
      \item No se deben poner tildes.
      \item La sopa se separa con espacios.
    \end{enumerate}


  \section{Conclusión}




\end{document}
